% % References:
% % schopp: https://ulrichschoepp.de/Docs/th.pdf
% % biering: https://ncatlab.org/nlab/files/Biering-BunchedLogic.pdf
% % blechschmidt: https://rawgit.com/iblech/internal-methods/master/paper-qcoh.pdf

\documentclass{article}

% https://ncatlab.org/nlab/show/Yoneda+embedding#ReferencesNotation
\usepackage[utf8]{inputenc}
\DeclareFontFamily{U}{min}{}
\DeclareFontShape{U}{min}{m}{n}{<-> udmj30}{}
\newcommand\yo{\!\text{\usefont{U}{min}{m}{n}\symbol{'207}}\!}

\usepackage{amsmath}
\usepackage{amssymb}
\usepackage{amsthm}
\usepackage{mathpartir}
\usepackage{quiver}

\usepackage{todonotes}

\newtheorem{theorem}{Theorem}[section]
\newtheorem{lemma}[theorem]{Lemma}
\newtheorem{example}[theorem]{Example}

\newcommand*{\Set}{\mathbf{Set}}
\newcommand*{\Ring}{\mathbf{Ring}}
\newcommand*{\CRing}{\mathbf{CRing}}
\newcommand*{\Graph}{\mathbf{Graph}}
\newcommand*{\Sh}{\mathrm{Sh}}
\newcommand*{\Psh}{\mathrm{Psh}}
\newcommand*{\T}{\mathbb{T}}
\newcommand*{\B}{\mathbb{B}}
\newcommand*{\Z}{\mathbb{Z}}
\newcommand*{\C}{\mathcal{C}}

% https://tex.stackexchange.com/a/223246
\newcommand*{\wand}{\mathbin{-\mkern-6mu*}}

\title{Instantiating Bunched Type Theory for Monoidal Classifying Toposes}
\author{}

\begin{document}
\maketitle


\begin{abstract}
    I provide a simple condition for the classifying topos of a theory
    to be interpretable in the bunched type theory of \cite{schopp}.
    This provides a natural way for the internal logic to work with a monoidal structure on the topos.
    Examples include the Schanuel topos, which was explored extensively in \cite{schopp},
    and the classifying topos of noncommutative rings, which yields some sort of noncommutative synthetic differential geometry.
\end{abstract}

\listoftodos
\tableofcontents

\section{Introduction}

Every topos comes with an associated \emph{internal language}.
Each is built on intuitionistic logic, yet each has its own nature.
These logics can be studied for their own sake (e.g. \cite{sdg}),
or to learn more about the toposes they're built from (e.g. \cite{blechschmidt-thesis}).

Sometimes, the topos we're interested has an important monoidal product, distinct from both the product and coproduct.
So we would like the internal logic to likewise support this monoidal product.
While we could simply add axioms to the logic, saying this monoidal structure exists,
it's much more natural to move to a logical system that has the monoidal structure built in.
Such a thing is called \emph{bunched logic}.

In particular, \cite{schopp} introduces a bunched dependent type theory, which has the primitives we need.
But \cite{schopp} only applies their type theory to one particular case: the Schanuel topos.
If we want to apply it more generally, we need to prove a number of properties of our topos.

I streamline this process. Specializing to classifying toposes,
this paper provides an easy-to-check condition on a theory \(\T\)
that suffices to interpret \cite{schopp}'s type theory
in the classifying topos \(\Set[\T]\).

\section{Main Result}

\subsection{Setup}

Let \(\T\) be a geometric theory.

Suppose that for each pair \(A,B\) of sorts of \(\T\),
we have a relation \(\sim : A \times B \to \mathbf{Prop}\), geometrically defined in terms of \(\T\)'s signature.
For instance, if \(\T\) is the theory of rings, then \(\sim : R \times R \to \mathbf{Prop}\)
might be defined as \(a \sim b \iff ab = ba\).

For each sort \(A\), define the shorthand \(A_{\sim x} = \{a : A \mid a \sim x\}\).

We want to require these subsets to themselves form a \(\T\)-model.
If \(\T\) is the theory of rings, for instance, we want \(R_{\sim x}\) to be a ring, for any \(x : R\).

To state this precisely, say that \(\sim\) \emph{cuts out a sub-\(\T\)-model} if the following two conditions hold.
\begin{itemize}
    \item For any function symbol \(f : A_1 \times \dots \times A_n \to B\) and any \(x : X\),
    the restriction \(f_{\sim x} : (A_1)_{\sim x} \times \dots \times (A_n)_{\sim x} \to B_{\sim x}\) is well-defined for all \(x\).
    \item For any \(x : X\), every axiom of \(\T\) continues to hold when each sort \(A\) is replaced by \(A_{\sim x}\), each function symbol \(f\) is replaced by \(f_{\sim x}\), and each relation symbol is replaced by its restriction.
\end{itemize}

Let's look at the example where \(\T\) is the theory of \emph{local} rings,
where we define \(a \sim b \iff ab = ba\).

The first condition expands to the following five sequents.
\begin{align*}
    x : R, a : R, ax = xa, b : R, bx = xb &\vdash (a+b)x = x(a+b)
    \\ x : R, a : R, ax = xa, b : R, bx = xb &\vdash (ab)x = x(ab)
    \\ x : R, a : R, ax = xa &\vdash (-a)x = x(-a)
    \\ x : R &\vdash 0x = x0
    \\ x : R &\vdash 1x = x1
\end{align*}
In other words, it says that \(R_{\sim x} = \{a : R \mid a \sim x\}\) is a subring of \(R\), for all \(x : R\).

The second condition generates one requirement per axiom in the theory, but most of these are trivial.
The interesting one comes from the following axiom of local rings.
\begin{align*}
    a : R, b : R, (\exists c : R, (a+b)c = 1) &\vdash (\exists c : R, ac = 1) \lor (\exists c : R, bc = 1)
\end{align*}

The requirement generated is as follows.
\begin{align*}
    &x : R, a : R, ax = xa, b : R, bx = xb, (\exists c : R, cx = xc \land (a+b)c = 1)
    \\ &\vdash (\exists c : R, cx = xc \land ac = 1) \lor (\exists c : R, cx = xc \land bc = 1)
\end{align*}
This is exactly the condition that the subring \(R_{\sim x} \subseteq R\) is local.

\subsection{Theorem}

Without further ado, here is the main theorem of this paper.

\begin{theorem} \label{main}
    Let \(\T\) be a geometric theory.
    For each pair of sorts \(A,B\) of \(\T\), let \(\sim : A \times B \to \mathbf{Prop}\) be a geometrically-defined relation which cuts out a sub-\(\T\)-model.
    Further suppose \(\sim\) is symmetric; we have \(a : A, b : B, a \sim b \vdash b \sim a\) for any sorts \(A\) and \(B\).
    Then we can interpret the type theory \(\mathbf{BT}(*, 1, \Sigma, \Pi, \Pi^*)\), as defined in \cite[Section 5.1]{schopp}, in \(\Set[\T]\).
\end{theorem}

\subsection{Examples}

\begin{enumerate}
    \item 
    Consider the theory of rings, with \(a \sim b \iff ab = ba\).
    This is clearly symmetric and geometrically defined, and it cuts out a submodel
    because \(\{a : R \mid ax = xa\}\) is a subring of \(R\) for any \(x : R\).

    So Theorem \ref{main} applies, and we can interpret \(\mathbf{BT}(*, 1, \Sigma, \Pi, \Pi^*)\) in \(\Set[\Ring]\).

    This is the example that motivated me to write this paper.

    \item
    Let \(\T\) be the theory of infinite decidable objects.
    Its classifying topos is known as the Schanuel topos.
    \begin{align*}
        &N \text{ sort}
        \\ a : N, b : N &\vdash a \mathrel{\#} b  \text{ prop}
        \\ a : N, a \mathrel{\#} a &\vdash \bot
        \\ a : N, b : N &\vdash a = b \lor a \mathrel{\#} b
        \\ a_1 : N, \dots, a_n : N &\vdash \exists b : N, \bigwedge_i a_i \mathrel{\#} b
    \end{align*}
    Choose \(\sim\) to be the apartness relation: \(a \sim b \iff a \mathrel{\#} b\).
    This is symmetric and geometrically defined. And since \(N\) is an infinite decidable object,
    so is \(\{a : N \mid a \mathrel{\#} x\}\).

    So we can interpret \(\mathbf{BT}(*, 1, \Sigma, \Pi, \Pi^*)\) in the Schanuel topos.
    This is the example discussed in \cite{schopp}.

    \item
    For any theory, there is a trivial example where \(a \sim b\) always holds.
    So we can interpret \(\mathbf{BT}(*, 1, \Sigma, \Pi, \Pi^*)\) in any topos.
    This turns out to be the special case where the monoidal product coincides with the cartesian product.

    \item There seems to be a ``generic example'' in the theory of undirected graphs.
    \begin{align*}
        &V \text{ sort}
        \\ a : V, b : V &\vdash a \sim b \text{ prop}
        \\ a : V, b : V, a \sim b &\vdash b \sim a
    \end{align*}
\end{enumerate}

\section{Proof}

We begin by bringing in some results from \cite{schopp}.

\begin{lemma}
    The type theory \(\mathbf{BT}(*, 1, \Sigma, \Pi, \Pi^*)\) can be modeled in any \\
    \((*, 1, \Sigma, \Pi, \Pi^*)\)-type-category.
\end{lemma}

\begin{proof}
    \cite{schopp}, section 6.
\end{proof}

\begin{lemma}
    Let \(\B\) be a topos equipped with a strict affine symmetric monoidal closed structure \((*,\wand)\),
    such that \(- * A\) preserves pullbacks for all \(A\).
    Further suppose we have a canonical choice of pullback for every span in \(\B\).
    Then the family fibration over \(\B\) is a \((*, 1, \Sigma, \Pi, \Pi^*)\)-type-category.
\end{lemma}

\begin{proof}
    \cite{schopp}, section 3.5. Since we are in a topos, the condition that the monomorphism \(A * B \hookrightarrow A \times B\) is strong is trivial.
\end{proof}

Next, we specialize to sheaf toposes.

\begin{lemma}
    Let \((C,J)\) be a (small) site, such that \(C\) is finitely complete.
    In fact, suppose we have a canonical choice for pullbacks in \(C\).
    Let \(*\) be an affine symmetric monoidal structure on \(C\) that preserves pullbacks and covers.
    Then \(\Sh(C,J)\) satisfies the conditions of the above lemma.
\end{lemma}

\begin{proof}
    \(\Sh(C,J)\) is a Grothendieck topos, hence a topos.

    \vspace*{1em}\hrule\vspace*{1em}

    Theorem 4.3.2 from \cite{biering} says we get a monoidal structure \(\otimes^\Sh\) on \(\Sh(C,J)\)
    by transporting Day convolution across the adjunction \((\mathbf{a} \dashv i) : \Psh(C,J) \rightleftarrows \Sh(C,J)\).
    Its monoidal unit is the sheafification of the Yoneda embedding of \(*\)'s unit, which simplifies to the terminal object.
    Chasing definitions quickly shows \(\otimes^\Sh\) is symmetric.
    And corollary 4.3.10 from \cite{biering} says this monoidal structure is closed.

    In other words, \(\otimes^\Sh\) is an affine symmetric monoidal closed structure.
    We must show this is strict --- that the canonical map \(A \otimes^\Sh B \to A \times B\) is always a monomorphism.

    But, letting \(\otimes^\Psh\) represent Day convolution on \(\Psh(C)\),
    this map is just \(iA \otimes^\Psh iB \to iA \times iB\), restricted to sheaves.
    Expanding both \(\otimes^\Psh\) and \(\times\) as Day convolutions,
    this is a map, natural in \(c \in C^\mathrm{op}\), of the following type.
    \begin{scriptsize}\[\left(\int^{c_1,c_2 \in C} A(c_1) \times B(c_2) \times (c \xrightarrow{C} c_1 * c_2)\right) \to \left(\int^{c_1,c_2 \in C} A(c_1) \times B(c_2) \times (c \xrightarrow{C} c_1 \times c_2)\right)\]\end{scriptsize}

    This map is exactly what you'd expect it to be; it composes the \(c \xrightarrow{C} c_1 * c_2\) component
    with the canonical map \(c_1 * c_2 \xrightarrow{C} c_1 \times c_2\), and leaves everything else alone.
    So since that canonical map is mono, this is too.

    \vspace*{1em}\hrule\vspace*{1em}

    Finally, we have a canonical construction for pullbacks in \(\Sh(C,J)\), since we have one for \(C\).
\end{proof}

And finally, we set the site to a syntactic site, to understand classifying toposes.

\begin{lemma}
    Let \(\T\) be a geometric theory.
    For each pair of sorts \(A,B\) of \(\T\), let \(\sim : A \times B \to \mathbf{Prop}\) be a symmetric, geometrically-defined relation, which cuts out a sub-\(\T\)-model.
    Then the syntactic site for \(\T\) satisfies the conditions of the above lemma.
\end{lemma}

\begin{proof}
    Let us begin with a few shorthands.

    We'll allow ourselves to write sequents involving compound types, such as \(p : A \times B \vdash \pi_1 p \sim \pi_2 p\).
    These can be straightforwardly ``compiled out'' to ordinary sequents, such as \(a : A, b : B \vdash a \sim b\).

    Next, if we have a context \(\Gamma = (a_1 : A_1, \dots, a_n : A_n, \phi_1 \dots \phi_p)\),
    we'll reuse the name \(\Gamma\) for the type \(\{(a_1, \dots, a_n) : A_1 \times \dots \times A_n \mid \phi_1 \land \dots \land \phi_p\}\).
    If we have a substitution \(f : \Gamma \to \Delta\), there is then a natural way to define the term \(\gamma : \Gamma \vdash f\gamma : \Delta\).

    Finally, if \(\gamma = (a_1, \dots, a_n) : \Gamma\) and \(\delta = (b_1, \dots, b_m) : \Delta\),
    we write \(\gamma \sim \delta\) as a shorthand for \(\bigwedge_{i=1}^m \bigwedge_{j=1}^n a_i \sim b_j\).

    With that out of the way, let's begin the proof.

    \vspace*{1em}\hrule\vspace*{1em}

    First of all, the syntactic category is guaranteed to be finitely complete, with a canonical construction for pullbacks.
    Specifically, the empty context is terminal,
    and the pullback of the span \(\Gamma \xrightarrow{f} \Xi \xleftarrow{g} \Delta\) is the context \((\gamma : \Gamma, \delta : \Delta, f\gamma = g\delta)\).

    Second, we describe the monoidal structure. Given contexts \(\Gamma\) and \(\Delta\),
    define \(\Gamma * \Delta = (\gamma : \Gamma, \delta : \Delta, \gamma \sim \delta)\).

    Functoriality of \(*\) reduces to the fact that \(\sim\) respects substitutions, which further reduces to the fact that \(\sim\) respects function symbols.
    Symmetry reduces to the symmetry of \(\sim\). Choosing the empty context as monoidal unit, the unit laws hold on the nose, and associativity up to a rearrangement of contexts.
    All the coherences hold.
    And since \(\Gamma * \Delta\) is just \(\Gamma \times \Delta\) with extra propositional hypotheses,
    the map \(\Gamma * \Delta \hookrightarrow \Gamma \times \Delta\) is a monomorphism.
    Putting that all together, \(*\) is a strict affine symmetric monoidal product on the syntactic category.

    Third, we show that starring preserves pullbacks.
    
    Say we have a span \(\Gamma \xrightarrow{f} \Xi \xleftarrow{g} \Delta\), and another context \(\Theta\).
    The pullback of the span is the context \((\gamma : \Gamma, \delta : \Delta, f\gamma = g\delta)\).
    Starring with \(\Theta\) yields \((\gamma : \Gamma, \delta : \Delta, f\gamma = g\delta, \theta : \Theta, \gamma \sim \theta, \delta \sim \theta)\).
    
    On the other hand, if we star with \(\Theta\) first, we get the span
    \((\gamma : \Gamma, \theta : \Theta, \gamma \sim \theta) \xrightarrow{f * \mathrm{id}_\Theta} (\xi : \Xi, \theta : \Theta, \xi \sim \theta) \xleftarrow{g * \mathrm{id}_\Theta} (\delta : \Delta, \theta : \Theta, \delta \sim \theta)\).
    Taking the pullback yields \((\gamma : \Gamma, \theta_1 : \Theta, \gamma \sim \theta_1, \delta : \Delta, \theta_2 : \Theta, \delta \sim \theta_2, f\gamma = g\delta, \theta_1 = \theta_2)\),
    which is equivalent to what we got before.

    And fourth, we show that starring preserves covers.
    
    Suppose \(\{f_i : \Gamma_i \to \Delta \mid i \in I\} \in J\).
    This means \(\delta : \Delta \vdash \bigvee_{i \in I} \exists \gamma : \Gamma_i, f_i\gamma = \delta\) is provable.

    By the final hypothesis, the proof is still valid if we, throughout the proof, replace each sort \(X\) with \(X_{\sim \theta} = \{x : X \mid x \sim \theta\}\), and each function \(f\) with its restriction \(f_{\sim \theta}\).
    This is only stated when \(\theta : \Theta\) for some \emph{sort} \(\Theta\),
    but the same is true for any \emph{context} \(\Theta\),
    simply by iterating once for each variable in \(\Theta\).

    So \(\theta : \Theta, \delta : \Delta_{\sim \theta} \vdash \bigvee_{i \in I} \exists \gamma : (\Gamma_i)_{\sim \theta}, f_i\gamma = \delta\).
    But this is equivalent to the statement \(\delta : (\Delta * \Theta) \vdash \bigvee_{i \in I} \exists \gamma : (\Gamma_i * \Theta), (f_i * \mathrm{id}_\Theta)\gamma = \delta\),
    so \(\{f_i * \mathrm{id}_\Theta : \Gamma_i * \Theta \to \Delta * \Theta \mid i \in I\}\) is a cover, as required.
\end{proof}

Putting this all together:

\begin{proof}[Proof of Theorem \ref{main}]
    Combine the above four lemmas.
\end{proof}

\section{Using the Monoidal Structure}

It's not enough to interpret the type theory in our topos of choice.
We need some axioms to build on, telling us how the new monoidal structure behaves.
After all, there's no rule in the type theory to say \(*\) isn't the same thing as \(\times\)!

\begin{theorem}
    Recall that \(\Set[\T]\) contains the generic model of \(T\).
    So for any sort \(X\) in \(\T\), let \(U_X\) be the object of \(\Set[\T]\) that models it.
    And for any sorts \(X\) and \(Y\), let \(\sim_U : U_X \times U_Y \to \mathbf{Prop}\) be the internal relation modelling \(\sim\).
    Then, in the internal logic, the image of the inclusion \(U_X * U_Y \hookrightarrow U_X \times U_Y\)
    is \(\{(x,y) \mid x \sim_U y\}\).
\end{theorem}

In other words, in the Schanuel topos, \(N * N \simeq \{(x,y) : N^2 \mid x \mathrel{\#} y\}\).
In \(\Set[\Ring]\), \(R * R \simeq \{(x,y) : R^2 \mid xy = yx\}\).
% In \(\Set[\Graph]\), \(V * V\) is the set of edges of the generic graph. issue: duplicated

\begin{proof}
    \(U_X\) is given by the sheafification of the Yoneda embedding of the context containing a single variable of type \(x\).
    In short, \(U_X = \mathbf{a}\yo X\). Similarly for \(U_Y\). So we calculate, using Lemma 4.3.1 from \cite{biering}:
    \begin{align*}
        U_X * U_Y &= \mathbf{a}\yo X * \mathbf{a}\yo Y
        \\ &= \mathbf{a}(i\mathbf{a}\yo X \otimes_\mathrm{Day} i\mathbf{a}\yo Y)
        \\ &= \mathbf{a}(\yo X \otimes_\mathrm{Day} i\mathbf{a}\yo Y)
        \\ &= \mathbf{a}(\yo X \otimes_\mathrm{Day} \yo Y)
        \\ &= \mathbf{a}(c \mapsto \int^{c_1,c_2} (\yo X)c_1 \times (\yo Y)c_2 \times (c \xrightarrow{C} c_1 * c_2))
        \\ &= \mathbf{a}(c \mapsto \int^{c_1,c_2} (c_1 \xrightarrow{C} X) \times (c_2 \xrightarrow{C} Y) \times (c \xrightarrow{C} c_1 * c_2))
        \\ &= \mathbf{a}(c \mapsto (c \xrightarrow{C} X * Y))
        \\ &= \mathbf{a}(\yo (X * Y))
    \end{align*}

    This is the internalization of the context \(X * Y = (x : X, y : Y, x \sim y)\),
    so it equals \(\{(x : U_X, y : U_Y) \mid x \sim_U y\}\).
\end{proof}

This shows what happens when we monoidally combine types taken from the generic model.
It's also interesting consider the opposite. What happens if we combine types that don't mention the generic model at all?
\begin{theorem}
    Let \(A, B \in \Set[\T]\), where \(A\) is the inverse image of a set \(\mathcal{A}\) under the global sections functor.
    Then \(A * B \simeq A \times B\).
\end{theorem}
So, for example, \(\mathbf{2} * X \simeq \mathbf{2} \times X \simeq X + X\).

\begin{proof}
    We have the following chain of isomorphisms:
    \begin{small}
    % https://q.uiver.app/#q=WzAsOSxbMCwwLCJcXGhvbShBICogQiwgWCkiXSxbMCwxLCJcXGhvbShBLCBCIFxcd2FuZCBYKSJdLFswLDIsIlxcaG9tXFxsZWZ0KFxcbGVmdChcXGNvcHJvZFxcbGltaXRzX3thIFxcaW4gXFxtYXRoY2Fse0F9fSBcXG1hdGhiZnsxfVxccmlnaHQpLCBCIFxcd2FuZCBYXFxyaWdodCkiXSxbMCwzLCJcXHByb2RcXGxpbWl0c197YSBcXGluIFxcbWF0aGNhbHtBfX0gXFxob20oXFxtYXRoYmZ7MX0sIEIgXFx3YW5kIFgpIl0sWzEsMywiXFxwcm9kXFxsaW1pdHNfe2EgXFxpbiBcXG1hdGhjYWx7QX19IFxcaG9tKEIsIFgpIl0sWzIsMywiXFxwcm9kXFxsaW1pdHNfe2EgXFxpbiBcXG1hdGhjYWx7QX19IFxcaG9tKFxcbWF0aGJmezF9LCBCIFxcdG8gWCkiXSxbMiwyLCJcXGhvbVxcbGVmdChcXGxlZnQoXFxjb3Byb2RcXGxpbWl0c197YSBcXGluIFxcbWF0aGNhbHtBfX0gXFxtYXRoYmZ7MX1cXHJpZ2h0KSwgQiBcXHRvIFhcXHJpZ2h0KSJdLFsyLDEsIlxcaG9tKEEsIEIgXFx0byBYKSJdLFsyLDAsIlxcaG9tKEEgXFx0aW1lcyBCLCBYKSJdLFswLDEsIiIsMCx7InN0eWxlIjp7InRhaWwiOnsibmFtZSI6ImFycm93aGVhZCJ9fX1dLFsxLDIsIiIsMCx7InN0eWxlIjp7InRhaWwiOnsibmFtZSI6ImFycm93aGVhZCJ9fX1dLFsyLDMsIiIsMCx7InN0eWxlIjp7InRhaWwiOnsibmFtZSI6ImFycm93aGVhZCJ9fX1dLFszLDQsIiIsMCx7InN0eWxlIjp7InRhaWwiOnsibmFtZSI6ImFycm93aGVhZCJ9fX1dLFs0LDUsIiIsMCx7InN0eWxlIjp7InRhaWwiOnsibmFtZSI6ImFycm93aGVhZCJ9fX1dLFs1LDYsIiIsMCx7InN0eWxlIjp7InRhaWwiOnsibmFtZSI6ImFycm93aGVhZCJ9fX1dLFs2LDcsIiIsMCx7InN0eWxlIjp7InRhaWwiOnsibmFtZSI6ImFycm93aGVhZCJ9fX1dLFs3LDgsIiIsMCx7InN0eWxlIjp7InRhaWwiOnsibmFtZSI6ImFycm93aGVhZCJ9fX1dXQ==
    \[\begin{tikzcd}
        {\hom(A * B, X)} && {\hom(A \times B, X)} \\
        {\hom(A, B \wand X)} && {\hom(A, B \to X)} \\
        {\hom\left(\left(\coprod\limits_{a \in \mathcal{A}} \mathbf{1}\right), B \wand X\right)} && {\hom\left(\left(\coprod\limits_{a \in \mathcal{A}} \mathbf{1}\right), B \to X\right)} \\
        {\prod\limits_{a \in \mathcal{A}} \hom(\mathbf{1}, B \wand X)} & {\prod\limits_{a \in \mathcal{A}} \hom(B, X)} & {\prod\limits_{a \in \mathcal{A}} \hom(\mathbf{1}, B \to X)}
        \arrow[tail reversed, from=1-1, to=2-1]
        \arrow[tail reversed, from=2-1, to=3-1]
        \arrow[tail reversed, from=2-3, to=1-3]
        \arrow[tail reversed, from=3-1, to=4-1]
        \arrow[tail reversed, from=3-3, to=2-3]
        \arrow[tail reversed, from=4-1, to=4-2]
        \arrow[tail reversed, from=4-2, to=4-3]
        \arrow[tail reversed, from=4-3, to=3-3]
    \end{tikzcd}\]
    \end{small}

    So the result follows by Yoneda.
\end{proof}

\section{Applications}

The main purpose of this paper is to enable the exploration of monoidal toposes via their internal logic.

\subsection{Schanuel Topos, and Variations} \label{section-schanuel-and-variations}

\cite{schopp} does a good job of this for the Schanuel topos, the classifying topos of infinite decidable objects.
They focus on modeling the syntax of programming languages, using the generic infinite decidable object as their type of abstract variables.

Our formulation suggests an interesting extension to this idea.
Instead of infinite decidable \emph{objects}, with \(\sim\) being apartness,
what if we work with unbounded dense decidable \emph{linear orders}, with \(\sim\) being the strict inequality?
We lose symmetry, so we'd need to modify the type theory to make \(*\) asymmetric.
But now our abstract variables come with a natural ordering relation, where \(A * B\)
means all the variables in \(A\) come \emph{before} all the variables in \(B\).

Is this useful? I have no idea. Is it interesting? Definitely!

\subsection{Noncommutative Synthetic Differential Geometry}

I was inspired to write this paper by a use-case related to \emph{synthetic differential geometry}.

Synthetic differential geometry (\cite{sdg}) is the study of toposes whose internal logics satisfy, at minimum, the following axioms.
\begin{itemize}
    \item There is a commutative ring, \(R\).
    \item Let \(D = \{x \in R | x^2 = 0\}\). Then the map \(R^2 \to (D \to R)\) given by \((a,b) \mapsto (d \mapsto a + bd)\) is an isomorphism.
\end{itemize}

This allows the immediate definition of derivatives, tangent vectors, or other concepts from differential geometry.
For instance, if \(f : R \to R\), \(f'\) is simply the unique function satisfying \(f(x+d) = f(x) + f'(x) d\) for all \(x : X\) and \(d : D\).
Limits and complicated epsilon-delta arguments are completely sidestepped; \emph{every} function \(R \to R\) is smooth.

A prototypical, though suboptimal, topos satisfying these axioms is the classifying topos of commutative rings.
But this immediately suggests a \emph{noncommutative} version of synthetic differential geometry, where we work with \(\Set[\Ring]\) instead of \(\Set[\CRing]\).

When I tried this, the first thing I discovered is that \(\Set[\Ring]\) has an important monoidal structure,
given by Day convolution over the tensor product of rings.
This is exactly the monoidal structure we get if we set \(a \sim b = (ab = ba)\) in Theorem \ref{main}.

And indeed, \(\Set[\Ring]\) supports a \emph{monoidal} version of the synthetic differential geometry axioms.
\begin{theorem} \label{nsdg}
    Internally to \(\Set[\Ring]\), let \(R\) be the generic ring, and let \(D = \{x \in R | x^2 = 0\}\).
    Then \(R^2 \simeq (D \wand R)\).
\end{theorem}


\begin{proof}
    We reason analogously to the proof in section 9.3 of \cite{sgm}.

    Recall that \(\Set[\Ring]\) is the topos of presheaves on the opposite of the category of finitely presented rings.
    We have \(R = \yo(\Z[x])\), \(D = \yo(\Z[d]/(d^2))\), and \(R^2 = \yo(\Z[a,b])\).

    We want to show \(R^2\) is the monoidal exponential \(D \wand R\).
    So since monoidal exponentials are preserved by Yoneda,
    it suffices to show that \(\Z[a,b]\) is a monoidal exponential of \(\Z[x]\) and \(\Z[d]/(d^2)\) in the opposite of the category of finitely presented rings.
    That is, that it's the monoidal \emph{coexponential} in finitely presented rings.

    We check the universal property directly.
    
    Define the coevaluation map \(\epsilon \in \hom_{\Ring_\mathrm{fp}}(\Z[x], \Z[a,b] \otimes \Z[d]/(d^2))\)
    to be the unique map taking \(x\) to \(a \otimes 1 + b \otimes d\).

    Now, every \(f \in \hom_{\Ring_\mathrm{fp}}(\Z[x], X \otimes \Z[d]/(d^2))\) is determined by where it maps \(x\).
    And \(f(x)\) will always have the form \(\alpha \otimes 1 + \beta \otimes d\), for some unique choice of \(\alpha, \beta \in X\).
    But then \(f = (g \otimes \mathrm{id}) \circ \epsilon\),
    where \(g\) maps \(a\) to \(\alpha\) and \(b\) to \(\beta\).

    So every \(f \in \hom_{\Ring_\mathrm{fp}}(\Z[x], X \otimes \Z[d]/(d^2))\) factors through \(\epsilon\).
    And this factorization is unique, because every possible \(g\) is hit by the procedure just described.
\end{proof}

A final point of interest: Internally to \(\Set[\CRing]\), the type \(R \to R\) is exactly the space of polynomials.
But when we move to \(\Set[\Ring]\) and \(R\) becomes noncommutative, there are now \emph{two} reasonable notions of polynomial.
Either the coefficients are assumed to commute with the indeterminate, or they aren't.
These two notions correspond exactly to the two choices of function space: \(R \wand R\) and \(R \to R\).

\subsection{Separation Logic}

A more speculative application is \emph{separation logic},
a method of proving the correctness of imperative programs.

Traditionally, separation logic is an intuitionistic logic, equipped with an extra monoidal closed structure.
To me, this looks like a \emph{decategorified} monoidal topos.
Might there be some clever choice of theory \(T\) and relation \(\sim\),
so that the interpretation of \(\mathbf{BT}(*, 1, \Sigma, \Pi, \Pi^*)\)
in the classifying topos \(\Set[\T]\) acts as a categorified separation logic?

\section{Future Work}

Besides the applications detailed above, I can see a few other worthwhile lines of research.

\subsection{Reasoning Principles}

When discussing the application to \(\Set[\Ring]\), the proof of Theorem \ref{nsdg}
stepped outside the internal logic, reasoning directly about the topos in its incarnation as a functor category.
I find this unsatisfying.

By contrast, when proving the version of this for \(\Set[\CRing]\),
we can begin with the axiom scheme provided by \cite{blechschmidt},
then proceed to work entirely within the internal logic.
Specifically, the axiom tells us that \(A \simeq (\hom_{R\text{-alg}}(A,R) \to R)\) for any \(R\)-algebra \(A\);
specializing to \(A = R[d]/(d^2)\) shows that \((D \to R) \simeq R[d]/(d^2)\), which is set-isomorphic to \(R^2\).

It would be valuable to either extend \cite{blechschmidt}'s axiom scheme to allow internal proofs of theorems like Theorem \ref{nsdg},
or show how to do so using the existing axiom scheme.

\subsection{Modifying the Type Theory}

It might also make sense to explore modifications or extensions to the type theory.

We saw this already in Section \ref{section-schanuel-and-variations},
where the theory of unbounded dense decidable linear orders wants \(*\) to be asymmetric.

It also seems that the \emph{relation} \(\sim : A \times B \to \mathbf{Prop}\) ought to be generalized to a \emph{dependent type} \(\sim : A \times B \to \mathbf{Type}\).
% and that this would correspond to the internal maps \(A * B \to A \times B\) no longer being injective.
What changes to \cite{schopp}'s type theory does this require, if any?

And finally, \cite{schopp} found it useful to add a monoidal \(\Sigma\) type to their type theory, analogous to the monoidal \(\Pi\) type.
What additional structure is needed on \(\T\) for the interpretation in \(\Set[\T]\) to support this?

\todo{References}
\todo{General proofreading}

\end{document}
